\documentclass[Main]{subfiles}
\begin{document}


\textbf{Optimising memory use}
The excessive memory use is rectified by creating temporary 2d-char-arrays to contain the level walls, boxes and goals, so the actual required size is known before the Node is created. The required size of the arrays are set in the Node constructor. 

The walls and goals of the ChildNode are just set to the walls and goals of the parent Node, which is perfectly fine as they are static. 

Regarding the boxes 

Optimising the memory use when regarding boxes



..... Something about the box structure not being fitting for the case, since there are very few boxes compared to the possible large 2-dimensional arrays they are stored in.  ......


\textbf{Benchmark with optimisations}

\begin{table}[h]
\begin{tabular}{llllll}
\rowcolor[HTML]{EFEFEF} 
\textbf{Level} & \textbf{Client} & \textbf{Time} & \textbf{Memory used} & \textbf{Solution length} & \textbf{Nodes explored} \\
SAD1           & BFS             & 0.01 s        & 5.12 MB              & 19                       & 78                      \\
SAD1           & DFS             & 0.01 s        & 5.12 MB              & 27                       & 44                      \\
SAD2           & BFS             & -           & -                  & -                      & -                     \\
SAD2           & DFS             & 1.23 s        & 23.10 MB             & 5781                     & 6799                    \\
custom         & BFS             & -           & -                  & -                      & -                     \\
custom         & DFS             & 0.01 s        & 5.12 MB              & 45                       &                        
\end{tabular}
\end{table}

\end{document}
