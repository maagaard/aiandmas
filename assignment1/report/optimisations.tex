\documentclass[Main]{subfiles}
\begin{document}


\textbf{Sub title}



% Please add the following required packages to your document preamble:
% \usepackage[table,xcdraw]{xcolor}
% If you use beamer only pass "xcolor=table" option, i.e. \documentclass[xcolor=table]{beamer}
\begin{table}[h]
\begin{tabular}{llllll}
\rowcolor[HTML]{EFEFEF} 
\textbf{Level} & \textbf{Client} & \textbf{Time} & \textbf{Memory used} & \textbf{Solution length} & \textbf{Nodes explored} \\
SAD1           & BFS             & 0.01 s        & 5.12 MB              & 19                       & 78                      \\
SAD1           & DFS             & 0.01 s        & 5.12 MB              & 27                       & 44                      \\
SAD2           & BFS             & DNF           & DNF                  & DNF                      & DNF                     \\
SAD2           & DFS             & 1.23 s        & 23.10 MB             & 5781                     & 6799                    \\
custom         & BFS             & DNF           & DNF                  & DNF                      & DNF                     \\
custom         & DFS             & 0.01 s        & 5.12 MB              & 45                       &                        
\end{tabular}
\end{table}



1) ????
2) The number of columns are used when instantiating the node
Benchmark:


b) ``The locations of boxes in a level are not static. Explain which data structure would allow you to save memory in most levels, while still offering good performance when it comes to lookup. In terms of running time, what would the impact of such a modification be on isGoalState() and getExpandedNodes()''

???

c)

d) 

Something about the box structure not being fitting for the case, since there are very few boxes compared to the possible large 2-dimensional arrays they are stored in. 

\end{document}
