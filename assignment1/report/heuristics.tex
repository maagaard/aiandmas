\documentclass[Main]{subfiles}
\begin{document}


\textbf{Sub title}



% Please add the following required packages to your document preamble:
% \usepackage[table,xcdraw]{xcolor}
% If you use beamer only pass "xcolor=table" option, i.e. \documentclass[xcolor=table]{beamer}
\begin{table}[h]
\begin{tabular}{llllll}
\rowcolor[HTML]{EFEFEF} 
\textbf{Level} & \textbf{Evaluation} & \textbf{Time} & \textbf{Memory used} & \textbf{Solution length} & \textbf{Nodes explored} \\
SAD1           & A*                  & 0.01 s        & 5.12 MB              & 19                       & 35                      \\
SAD1           & WA*                 & 0.01 s        & 5.12 MB              & 19                       & 19                      \\
SAD1           & Greed               & 0.01 s        & 5.12 MB              & 19                       & 19                      \\
SAD2           & A*                  & 0.01 s        & 5.12 MB              & 19                       & 19                      \\
SAD2           & WA*                 & 0.01 s        & 5.12 MB              & 19                       & 19                      \\
SAD2           & Greedy              & 0.01 s        & 5.12 MB              & 19                       & 19                      \\
custom         & A*                  & 0.01 s        & 5.12 MB              & 19                       & 19                      \\
custom         & WA*                 & 0.01 s        & 5.12 MB              & 19                       & 19                      \\
custom         & Greedy              & 0.01 s        & 5.12 MB              & 19                       & 19                      \\
FireFly        & A*                  & -           & -                  & -                      & -                     \\
FireFly        & WA*                 & -           & -                  & -                      & -                     \\
FireFly        & Greedy              & -           & -                  & -                      & -                     \\
Crunch         & A*                  & -           & -                  & -                      & -                     \\
Crunch         & WA*                 & -           & -                  & -                      & -                     \\
Crunch         & Greedy              & -           & -                  & -                      & -                     
\end{tabular}
\end{table}


a) ``Write a best-first search client by implementing the StrategyBestFirst class. The Heuristic argument in the constructor must be used to order nodes. As it implements the Comparator<Node> interface it integrates well with the Java Collections library. Make sure you use an appropriate data structure for the frontier, and state which one you have used.''


The frontier is a List. When a node (the one with the lowest f(n)) is poped from the frontier, the frontier is cleared in order to make a new frontier based on the chosen node. The remaining nodes in the frontier, before clearing it, is stored as an old frontier - in order to go back to that, if the new one proves to be a dead end. 

Actually I use two frontiers. 


b)
Heuristic function
The heuristic functions functions calculates the distance from the agent to the box and from the box to the goal. The goal and box coordinates are found in the pre-processing.

Issue when multiple of same box or goal - because it is stored in a map with character as key, where keys are unique. 


Benchmark:





c) 
Admissibility:

The heuristic function is admissible, as it calculates the Pythagorean distance between agent, box and goal without regarding any possible obstacles. 



d) Competition levels:






\end{document}